\documentclass[11pt,a4paper]{article}

\usepackage[slovene]{babel}
\usepackage[utf8x]{inputenc}
\usepackage{graphicx}

\pagestyle{plain}

\begin{document}
\title{Poročilo pri predmetu \\
Analiza podatkov s programom R}
\author{Študent FMF}
\maketitle

\section{Izbira teme}
Kot strasten navijač nogometnega kluba Arsenal F.C. sem za svojo temo analiziranja izbral Analiza igralcev nogometnega kluba Arsenal Football Club. Osnovna ideja je analiza vsakega igralca v zgodovini nogometnega kluba po različnih spremenljivkah treh tipov. Podatke iz spletne strani bom prenesel v program  Microsoft Office Excel Worksheet in tako oblikoval novo tabelo, ki se bo od originalne razlikovala še za dodano urejenostno spremenljivo, ki v originalni tabeli ni podana. Tabelo bom shranil v CSV obliki, tako da jo bom lahko uvozil v program  R. Za vsakega igralca, ki je kadarkoli zaigral za Arsenal F.C., bom podal: 
- državo iz katere igralec prihaja (imenska spremenljivka)
- njegovo standardno pozicijo v igri (imenska spremenljivka,ki bo podana v angleških,mednarodnih kraticah)
- letnice delovanja v omenjenem nogometnem klubu (številska spremenljivka, podana v obliki od-do)
- število nastopov za klub (številska spremenljivka)
- število zadetkov za klub (številska spremenljivka)
- status nogometaša v klubu (urejenostna spremenljivka, katero bom opredelil glede na število nastopov, ki jih je nogometaš dosegel v klubu. Nogometaše bom razdelil v tri skupine: "Začetnik v Arsenalu"(do 150 nastopov), "izkušenj Arsenalovec"(od 150-300 nastopov), "Arsenalova legenda"(nad 300 nastopov)).
Države, katerih državljani so bili in so še nogometaši bom prikazal tudi na zemljevidu.

Cilj projekta je ugotovitev katerim nogometašem se je uspelo prebiti v klubsko zgodovino s svojim doprinosom v klubu, predvsem pa spoznavanje orodja analiziranja v programu R na nekem konkretnem, zame osebno zanimivem primeru. 

Povezava do podatkovne tabele o moji tematiki: 
http://en.wikipedia.org/wiki/List_of_Arsenal_F.C._players

\section{Obdelava, uvoz in čiščenje podatkov}

\section{Analiza in vizualizacija podatkov}

\includegraphics{../slike/povprecna_druzina.pdf}

\section{Napredna analiza podatkov}

\includegraphics{../slike/naselja.pdf}

\end{document}
